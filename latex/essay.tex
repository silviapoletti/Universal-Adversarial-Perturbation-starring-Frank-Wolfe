\documentclass[10pt,twocolumn,letterpaper]{article}

\usepackage{cvpr}
\usepackage{times}
\usepackage{epsfig}
\usepackage{graphicx}
\usepackage{amsmath}
\usepackage{amssymb,bbm,xcolor}
\usepackage[breaklinks=true,bookmarks=false]{hyperref}
\usepackage{lipsum}
\usepackage{listings}
\usepackage{mathtools,eucal}
\usepackage{graphicx}
\graphicspath{ {../data/img/decentralized_stoch/} {../data/img/distributed/}{../data/img/variance_reduced/} {../data/perturbations/decentralized_stoch/}{../data/perturbations/variance_reduced/}{../data/perturbations/distributed/}{../data/img}}
\usepackage{bbold}
\usepackage{caption}
\usepackage{subcaption}
\usepackage{algorithm}
\usepackage[noend]{algpseudocode}
\usepackage{booktabs, multicol, xcolor}
\usepackage[shortlabels]{enumitem}
\usepackage{changepage}
\cvprfinalcopy % *** Uncomment this line for the final submission
\setcounter{page}{1}
\renewcommand{\figurename}{Figure}
\usepackage{amsmath}
\DeclareMathOperator*{\argmax}{arg\,max}
\DeclareMathOperator*{\argmin}{arg\,min}
\DeclareMathAlphabet\mathbfcal{OMS}{cmsy}{b}{n}

\begin{document}
\title{Universal Adversarial Perturbation \\ starring Frank-Wolfe}
\author{Chiara Bigarella\\{\tt\footnotesize Student nr. 2004248}\and Silvia Poletti\\{\tt\footnotesize Student nr. 1239133}\and Gurjeet Singh\\{\tt\footnotesize Student nr. 2004251}\and Francesca Zen\\{\tt\footnotesize Student nr. 2010640}}
\maketitle

% TODO: insert sections here 
\begin{abstract}
	The main goal of this report is to analyze three different Stochastic Gradient Free Frank-Wolfe algorithms for producing universal adversarial perturbations. These perturbations are designed to fool the LeNet-5 convolutional neural network considering the classification task on the MNIST dataset.\\
	\indent Before discussing the implementation, some key concepts about the adversarial attacks and the Frank-Wolfe framework are introduced. The last two sections instead concern the experiments and the conclusive comparison of the results.
\end{abstract}
\section{Introduction}
The main goal of this report is to analyze three different Strochastic Gradient Free Frank-Wolfe algorithms for producing universal adversarial perturbations. These perturbations are designed to fool the LeNet5 convolutional neural network considering the classification task on the MNIST dataset.\\
Before discussing the implementation, some key concepts about the adversarial attacks and the Frank-Wolfe framework are introduced. The last two sections concern the experiments and the conclusive comparison of the results.

\subsection{Adversarial Attacks}
An adversarial attack is a machine learning technique that has the aim of fooling a classifier by providing it with
carefully designed inputs, called \textit{adversarial examples}. An adversarial example is a datapoint that has
been perturbed or distorted in order to cause the classifier to make an erroneous prediction.

Adversarial attacks can be divided into targeted and untargeted attacks. \textit{Targeted} attacks have the aim to
misclassify the input image to a target class, while \textit{untargeted} attacks are aimed at only misclassifying
the input image to any other class except the true class.

A further way to categorize adversarial attacks is to divide them into white-box and black-box attacks, based on
the information accessible to the attacker. To perform a \textit{white-box} attack, the classifier, the trained
parameter values, and the analytical form of the classifier loss function must be all available to the attacker. On
the contrary, in a \textit{black-box} setting, only a zeroth-order oracle is accessible, that is for an input $(\mathbf{x},y)$,
the value of the loss function $F(\mathbf{x},y)$ is accessible. Therefore, black-box attacks can be seen as an optimization problem,
where the adversarial attack is formulated as a maximization of the classification loss function, whose values are
accessible only via a zeroth-order oracle.

This kind of adversarial attacks can be further categorized into score-based and decision-based attacks.
In \textit{score-based} attacks, the attacker directly observes a loss value, class probability, or some other
continuous output of the classifier on a given example, whereas in \textit{decision-based} attacks, the attacker gets to
observe only the hard label predicted by the classifier.

% For example, an adversarial example would be a perturbed image of a pig which would still be a pig to a human eye, while
% the classifier would now output the class to be that of an airplane.

% However, in most real-world deployments, it is impractical to assume complete access to the classifier
% and analytic form of the corresponding loss function, which makes black-box settings more realistic.

% Black-box adversarial attacks can be broadly categorized across a few different dimensions: optimization-based
% versus transfer-based attacks, and score-based versus decision-based attacks.
% The zeroth-order information can be further categorized into score-based and decision-based attacks.

In this report we will focus on untargeted, black-box, score-based adversarial attacks.

\subsection{Universal Adversarial Perturbations}
Universal perturbations are small perturbations that, when applied to the input images, are able
to fool a state-of-the-art deep neural network classifier. These perturbations are quasi-imperceptible to human eyes,
due to their small norm, and therefore they are difficult to detect. However, what makes universal perturbations special,
is their capability to generalize well both on a large dataset and across different deep neural network architectures.
In fact, an important property of universal perturbations is that they are image-agnostic. This means that
they don't depend on a single image, but rather they are able to cause label estimation change for most of
the images contained in a dataset.

It is important to notice that there's no unique universal perturbation: different random shuffling of the data used
to compute the perturbation can lead to a diverse set of universal perturbations.

Furthermore, universal perturbations mostly make natural images classified with specific labels, called
\textit{dominant labels}. These dominant labels are not determined a priori, but rather they are automatically
found by the algorithm that computes the universal perturbation. In paper \textcolor{red}{XXX}, the authors
hypothesize that "these dominant labels occupy large regions in the image space, and therefore represent good
candidate labels for fooling most natural images".

Finally, although some fine-tuning strategies can be adopted to make deep neural networks more robust to adversarial
attacks, they are not sufficient to make the classifiers immune to the attacks. In fact, the fine-tuning of a classifier
with universal perturbations leads to an improvement in the classification of perturbed images, however it is always
possible to find a small universal perturbation that can fool the network.

% non-convex problem

% the objective is not to find the smallest perturbation that fools most of the data points,
% but rather to find one such perturbation with sufficiently small norm

% universal perturbations exploit some geometric correlations between different parts in the decision
% boundary of the classifier


\section{Gradient Free Frank Wolfe}
Gradient free optimization algorithms find application in settings where the explicit closed form of the loss function is not available or the gradient evaluation is computationally prohibitive. A prime example is black-box adversarial attacks on neural networks, where only the model output is known while the architecture and weights remain unknown. In fact, gradient free methods exploits just zeroth-order oracle calls (i.e. loss function evaluations) to solve an optimization problem.\\
In particular, this report focuses on zeroth-order Frank Wolfe algorithms for constrained optimization problems. Unlike the Projected Gradient method that requires expensive projection operations, the Frank Wolfe framework provides computational semplicity by making use of instances of linear minimization. \\
Considering the application of interest, that is black-box adversarial attacks, a "good enough" feasible solution is often adequate. Therefore, the use of biased gradient estimates obtained from zeroth-order information is suitable for performing this task.\\

1) SPIEGARE ZEROTH-ORDER OPTIMIZATION + STOCHASTIC AND CONSTRAINED OPTIMIZATION\\
2) SPIEGARE FW FROM 1ST ORDER TO 0TH ORDER\\
3) INTRODURRE IL CONCETTO DI DECENTRALIZED E DISTRIBUTED SETTINGS\\


\section{Methods}
Recently, decentralized and distributed settings are getting significant attention due to the need to exploit data
parallelization in case of complex models trained on huge datasets, that typically require a storage that exceeds
a machine capacity. In brief, in decentralized setups the devices (or workers) exchange information with a master node, while in distributed setups the devices are connected in a peer-to-peer manner and therefore exchange information only with their neighbors. \\
\indent The following optimization algorithms are designed to accommodate distributed data across multiple devices.\\
\indent All the proposed algorithms will consider as iterates the perturbations \mbox{\boldmath$ \delta$}$_t$ converging to the solution of the optimization problem (1). In particular, the iterates $\mathbf{x}_t$ in the previously defined gradient approximation schemes have to be intended as a batch of dataset images $\mathbf{x}$ to which the t-th perturbation \mbox{\boldmath$ \delta$}$_t$ has been applied, i.e. $\mathbf{x}_t = \mathbf{x} + \mbox{\boldmath$ \delta$}_t$ are the perturbed data.
\subsection{Decentralized Stochastic Gradient Free Frank Wolfe}
In this section we discuss the Stochastic Gradient Free FW in a decentralized setup. This setting architecture is composed of a central node, called master node, and other nodes connected to it but no to each other, called workers.
Workers are connected to the master node to read, write and exchange information.\\
In our case we have $M$ workers to which is spread the data, they compute their local gradient using the loss function and the parameters given from I-RDSA. Then, they send the estimated gradients to the master node who uses them to return the new perturbed data.\\
In particular, given an input image \boldmath$\delta$, the algorithm initializes both the starting point and the gradient to a zero $d$-vector and $M \times d$ matrix, respectively. Then each worker computes its own gradient estimation based on the parameter $\rho_t$ given by I-RDSA in the following way:
\[\textbf{g}_i= (1- \rho_t)\textbf{g}_{i,t-1} + \rho_t\textbf{g}_i(\text{\boldmath}\delta_t,\textbf{y}).\]
 When all of the workers finish their tasks, they send the results to the master node, which computes the average of the estimated gradients and send back the obtained value as the new initial point \boldmath$\delta_{t+1}$:
\[\text{\boldmath}\delta_{t+1} = (1-\gamma_t)\text{\boldmath}\delta_t + \gamma_t\textbf{v}_t, \]
with $\gamma_t = \frac{2}{t+8}$. When all of the iterations are done, the algorithm returns the hystory of all the perturbations computed by the master node.

\begin{algorithm}
	\caption{Decentralized Stochastic Gradient Free FW}\label{decentralized} 
	 \textbf{Input} Input image \boldmath$\delta$, labels $y$, Loss Function $F(\text{\boldmath}\delta;y)$, number of queries $T$, number of workers $M$, image dimension $d$, tolerance $\varepsilon$, number of directions $m$.\\
	 \textbf{Output} Universal perturbation's history.
	\begin{algorithmic}[1]		
		\State Initialize \boldmath$\delta_0 = \text{0}$.
		\For {$t = 0, \dots, T-1$}
		\State Master node computes parameters required for the computation of the I-RDSA scheme: 
		{\scriptsize\[(\rho_t,c_t)_{I-RDSA} =\bigg(\frac{4}{\big(1+\frac{d}{m}\big)^{1/3}(t+8)^{2/3}}, \frac{2\sqrt{m}}{d^{3/2}(t+8)^{1/3}}\bigg)\]}
		\State For each worker $i$ compute I-RDSA:\newline Sample $\{\textbf{z}_{n,t}\}_{n=1}^m \sim N(0, \textbf{1}_d)$ \newline
		 $\textbf{g}_i(\text{\boldmath}\delta_t;\textbf{y}) = \frac{1}{m} \sum_{n=1}^{m} \frac{F(\text{\boldmath}\delta_t + c_t\textbf{z}_{n,t};\textbf{y}) - F(\text{\boldmath}\delta_t;\textbf{y})}{c_t}\textbf{z}_{n,t}$
		
		\State Workers compute \[\textbf{g}_i= (1- \rho_t)\textbf{g}_{i,t-1} + \rho_t\textbf{g}_i(\text{\boldmath}\delta_t,\textbf{y}).\]
		\State Push $\textbf{g}_{i,t}$ to the master node.
		\State Master node computes 
		\[\textbf{g}_t = \frac{1}{M} \sum_{i=1}^{M} \textbf{g}_{i,t}.\]
		\State Master node computes $\textbf{v}_t = - \varepsilon sign(\textbf{g}_t)$.
		\State Master node computes \boldmath$\delta_{t+1} = (1-\gamma_t)\text{\boldmath}\delta_t + \gamma_t\textbf{v}_t$ and sends it to all nodes.
		\EndFor

	\end{algorithmic}
\end{algorithm}

\subsection{Decentralized Variance-Reduced Stochastic Gradient Free Frank Wolfe}
This section analyzes the SPIDER variance reduction technique, which is built for dynamic tracking, while avoiding excessive querying to oracles and ultimately reducing query complexity.\\
At the beginning of Algorithm \ref{variance-reduced} we initialize our initial perturbation to a zero $d$ vector \boldmath$\delta_0$ and denote $q \in \mathbb{N}_{+}$ as a period parameter. At the beginning of each period, i.e. $mod(q,n)=0$, each worker employ KWSA for the computation of its own gradient estimation along the canonical basis vector $e_k$, with $k=1, \dots, d$, using $\eta = \frac{2}{d^{1/2}(t+8)^{1/3}}$. In all other cases, the worker selected a mini-batch $ S'$ of local component functions and use the RDSA to estimate and update the gradients. In particular, $S_2$ pairs of component functions are chosen for the computation of the gradient estimation
\[\textbf{e}_k^T\textbf{g}_i(\text{\boldmath} \delta_t) = \frac{1}{n}\sum_{j=1}^{n}\frac{F_{i,j}(\textbf{x}_t + \eta\textbf{e}_k) - F_{i,j}(\textbf{x}_t) }{\eta},\]
which is then sent to the master node. While running RDSA we use $\eta = \frac{2}{d^{3/2}(t+8)^{1/3}}$.\\
Then, the master node computes the average of the estimated gradients and calculates the new perturbation. Algorithm \ref{variance-reduced} returns all the perturbations computed by the master node.
\begin{algorithm}
	\caption{Decentralized Variance-Reduced Stochastic Gradient Free FW}\label{variance-reduced}
	\textbf{Input} Input image \boldmath$\delta$, labels $y$, Loss Function $F(\text{\boldmath}\delta;y)$, number of queries $T$, number of workers $M$, image dimension $d$, tolerance $\varepsilon$, number of images $S_1$, subset of component functions $S_2$, total number of component functions $n$, period $q$.\\
	\textbf{Output} Universal perturbation's history.
	\begin{algorithmic}[1]		
		\State Initialize \boldmath$\delta_0 = \text{0}$.
		\For {$t = 0, \dots, T-1$}
		\State Each worker $i$ computes:
		\If {$mod(t,q)=0$}
		\State Draw {\small$S_1' = \frac{S_1d}{M}$} samples for each dimension at each worker $i$ and compute its local gradient \newline
		{\small$ \textbf{e}_k^T\textbf{g}_i(\text{\boldmath} \delta_t) = \frac{1}{n}\sum_{j=1}^{n}\frac{F_{i,j}(\textbf{x}_t + \eta\textbf{e}_k) - F_{i,j}(\textbf{x}_t) }{\eta} $} along each canonical basis vector $\textbf{e}_k$.
		
		\State Each worker updates $\textbf{g}_{i,t} = \textbf{g}_i(x)_t$.
		\Else
		\State Draw $S_2$ pairs of component functions and Gaussian random vectors $\{\textbf{z}\}$ at each worker $i$ and update
		
		\parbox[b]{\linewidth}{$\textbf{g}_i(x)_t = \frac{1}{|S_2|} \sum_{j \in S_2}\frac{F_{i,j}(\textbf{x}_t + \eta\textbf{e}_k)- F_{i,j}(\textbf{x}_t) }{\eta} \textbf{z}$ -\\
			
			$\frac{F_{i,j}(\textbf{x}_{t-1} + \eta\textbf{e}_k) - F_{i,j}(\textbf{x}_{t-1}) }{\eta} \textbf{z}$}
				
		\State Each worker updates 
		\[\textbf{g}_{i,t} = \textbf{g}_i(\textbf{x}_t) \textbf{g}_{i,t-1}.\]
		\EndIf
		\State Each worker pushes $\textbf{g}_{i,t}$ to the master node.
		\State Master node computes 
		\[\textbf{g}_t = \frac{1}{M} \sum_{i=1}^{M} \textbf{g}_{i,t}.\]
		\State Master node computes $\textbf{v}_t = - \varepsilon sign(\textbf{g}_t)$.
		\State Master node computes \boldmath$\delta_{t+1} = (1-\gamma_t)\text{\boldmath}\delta_t + \gamma_t\textbf{v}_t$ and sends it to all nodes.
		\EndFor
	\end{algorithmic}
\end{algorithm}

\subsection{Distributed Stochastic Gradient Free Frank-Wolfe}
This section concerns the discussion of the Distributed Stochastic Grandient Free Frank-Wolfe method implemented in Algorithm 3 for the constraint optimization problem (7) in a distributed setup in which the $M$ workers do not have a central coordinator, instead they exchange information in a peer-to-peer manner. The internode communication network used by the workers is modeled as an undirected simply connected graph $G=(V,E)$, with $V=\{1, \dots, M\}$ the set of nodes and $E$ the set of comunication links. Each node communicates and exchanges information with its own neighbors. Given a node $n$, the set $\Omega_n = \{l \in V | (n,l)\in E\}$ indicates its neighborhood and $d_n = |\Omega_n|$ indicates its degree.\\ The $M \times M$ adjacency matrix $\mathbf{A}=[A_{ij}]$ describes the edges of the graph $G$: $A_{ij}=1$ if $(i,j) \in E$ and $A_{ij}=0$ otherwise. The diagonal matrix $\mathbf{D}=diag(d_1 \dots d_M)$ is used to compute the graph Laplacian $\mathbf{L}=\mathbf{D}-\mathbf{A}$. The normalized Laplacian $\mathbfcal{L} = [\mathcal{L}_{ij}]$ is define to be the matrix
\[
\mathcal{L}_{ij}=
\begin{cases}
	1 & \text{if $i=j$ and }d_i\ne0, \\
	-\frac{1}{\sqrt{d_id_j}} & \text{if $i$ and $j$ are adjacent,}\\
	0 & \text{otherwise.}
	
\end{cases}
\]
We can write $\mathbfcal{L} = \mathbf{D}^{-1/2}\mathbf{L}\mathbf{D}^{-1/2}$. The normalized Laplacian $\mathbfcal{L}$ is used to compute the weghted matrix $\mathbf{W} = \mathbb{1}- \mathbfcal{L}$.\\
Unlike the previous algorithms, the perturbation \mbox{\boldmath$ \delta$}$_0$ is initialized as a null $M \times d$ matrix, rather than a vector: this matrix is given in input to each worker, which only considers the rows corresponding to the perturbations computed by its neighbors. This trick simplifies the distributed architecture, without changing the functioning of the algorithm.\\
At every iteration, each worker $i$ exchanges its current iterate with its neighbors and averages the received iterates. Then, each worker $i$ applies the I-RDSA scheme to compute its local gradient estimation $\mathbf{g}_t^i$ and exchanges with the neighbors the vector $\mathbf{G}_t^i$, which is an averaged version of the gradient. Finally, when all the $\mathbf{G}_t^i$ have been calculated, they are involved in a weighted sum computed by each worker.\\
For what said until now, the implementation requires two round of communication: one for the iterates $\bar{\mathbf{x}}_t^i$ and one for the gradients $\bar{\mathbf{g}}_t^i$. The performance depends on how well these vectors are tracked across the network.\\
At the end of the t-th iteration, each worker computes the Frank-Wolfe update of the iterate \mbox{\boldmath$ \delta$}$_{t+1}^i$ using (5) on $\bar{\mathbf{g}}_t^i$ and (6) on $\bar{\mathbf{x}}_t^i$ with $\gamma_t = t^{-1/2}$.\\
\begin{algorithm}
	\caption{Distributed SGF FW}\label{distributed}
	\textbf{Input} Batches of images $\{\mathbf{x}_i\}_{i=1}^M$, batches of labels $\{\mathbf{y}_i\}_{i=1}^M$, loss function $F$, number of queries $T$, number of workers $M$, image dimension $d$, tolerance $\varepsilon$, number of sampled directions $m$, adjacency matrix $\mathbf{A}$, weight matrix $\mathbf{W}$.\\
	\textbf{Output} $\bar{\mbox{\boldmath$ \delta$}}_T^i\;\;\forall i \in \{1\dots M\}$
	\begin{algorithmic}[1]		
		\State For $i=1,\dots, M$ initialize \mbox{\boldmath$ \delta$}$_0^i =\mathbf{0} $ and $\bar{\mathbf{g}}_0^i = \mathbf{0}$
		\For {$t = 1, \dots, T$}
		\State Compute the parameter required for the computation of the I-RDSA schemes: 
		${\scriptsize c_t =\frac{2\sqrt{m}}{d^{3/2}(t+8)^{1/3}}}$
		\State In the first round of communication, each worker $i$ approximates the average iterate: \newline
		\[\bar{\mbox{\boldmath$ \delta$}}_t^i \leftarrow \sum_{j=1}^{M} W_{ij}\mbox{\boldmath$ \delta$}_t^i\]
		\State Each worker $i$ computes the I-RDSA scheme:\newline 
		Sample $\{\mathbf{z}_{j,t}\}_{j=1}^m \sim\mathcal{N}(0,\mathbb{1}_d)$ \newline
		\[\mathbf{g}_t^i = \frac{1}{m} \sum_{j=1}^{m} \frac{F(\mathbf{x}_i + \mbox{\boldmath$ \delta$}_t + c_t\mathbf{z}_{j,t}, \mathbf{y}_i) - F(\mathbf{x}_i + \mbox{\scriptsize\boldmath$ \delta$}_t, \mathbf{y}_i)}{c_t}\mathbf{z}_{j,t}\]
		
		\State  Each worker $i$ computes:
		\[ \mathbf{G}_t^i = \bar{\mathbf{g}}_{t-1}^i + \mathbf{g}_t^i - \mathbf{g}_{t-1}^i \]
		\State In the second round of communication, each worker $i$ approximates the average gradient:
		\[ \bar{\mathbf{g}}_t^i \leftarrow \sum_{j=1}^{M} W_{ij}\mathbf{G}_t^j  \]
		
		\State Each worker $i$ computes $\mathbf{v}_t^i = - \varepsilon sign(\bar{\mathbf{g}}^i_t)$.
		\State Each worker $i$ updates:
		\[\mbox{\boldmath$ \delta$}_{t+1}^i \leftarrow (1-\gamma_t)\bar{\mbox{\boldmath$ \delta$}}_t^i + \gamma_t\mathbf{v}_t^i\]
		\EndFor
		
	\end{algorithmic}
\end{algorithm}
\section{Experiments and results}
The data come from the MNIST dataset, containing 60000 train images and 10000 test images from 10 almost balanced classes (arabic numerals). The following simulations generate a universal adversarial perturbation starting from a portion of the MNIST train dataset. Then the perturbation is added to the test images that have been correctly classified by the pre-trained LeNet-5 convolutional neural network, aiming to maximally increase the loss function, and therefore minimizing the accuracy.\\
We now present the results of the three algorithm previously introduced: (i) Decentralized Stochastic Gradient Free Frank Wolfe, (ii) Decentralized Variance-Reduced Stochastic Gradient Free Frank Wolfe, (iii) Distributed Stochastic Gradient Free Frank Wolfe.\\
Our interest is to find a perturbation that is able to misclassify the predictions of LeNet-5 on MNIST digit images, by making also the noise on the images imperceptible to human eyes.\\ In order to generate an adversarial example we have to find $x'$ for a given input $x$ such that the corresponding loss function $L(x',y)$ is maximized, while minimizing the $\ell_{\infty}$ norm related to it. This minimization is needed for finding  the minimal perturbation that makes the digits unchanged. We set the $\delta$ noise within $\varepsilon=0.25$ by using the $\ell_{\infty}$ norm. We performed the experiments on the MNIST dataset with the aim to find an universal adversarial perturbation, and we defined a pretrained LeNet-5 model by using Keras/Tensorflow to test our algorithms and demonstrate the efficiency of the attacks.

\textcolor{gray}{Non so se ha senso metterlo qua, magari meglio discutere anche con Chiara se ha senso dire che comunque tutti gli algoritmi decentralize e distributed sono stati implementati in senquential mode invece che usare un vera e propria architettura distribuita. Dobbiamo far notare che comunque il codice e` stato implementato definendo i metodi in modo tale che sia scalabile facilmente alla modalita distribuita usando per esempio la libreria Ray}

\textcolor{gray}{Since the discussed algorithms have a decentralized or a distributed "part", we implemented them in a sequential manner but in such way that all the algorithms could be scaled and changed easily to a distributed or decentralized architecture. Indeed, all the methods have been implemented by dividing the workers and master nodes behaviour in different separate methods, and the extension to a distributed architecture could be reached by configuring a framework/library like Ray\footnote{Ray: https://github.com/ray-project/ray} or PySpark\footnote{PySpark: https://spark.apache.org/docs/latest/api/python/} for defining a distributed computing application.}

\subsection{Decentralized Stochastic Gradient Free Frank Wolfe}
\textcolor{blue}{Usiamo il passato ora, siccome e` un esperimento svolto}
To study the performance of Algorithm \ref{decentralized} we used the 10000 images in the MNIST test set, after normalizing them.\\ We split the digit images by giving 10 samples of each class to our 10 workers. In such way, we make each worker holding a hundred images. After tuning the hyperparameter $m$ of the number of direction, we discovered that $m=15$ was the good compromise between the computation time and the overall results. For each image we estimated its gradient using 20, 50 and 100 queries.

\textcolor{gray}{Accuracy error achieved, during the training of the model, the we need to compare the result achieved by testing the model on the whole test set.\\ Notice that the drop of the accuracy, hence the accuracy is discovered already at the 10 epoch/iteration of the algorithms. Comparing the three algorithm is almost the faster one, the competing algorithm in terms of speed convergence of the noise is the distributed.
\\ presence of the pattern in the noise when the accuracy is minimized, more is minimized more the noise the pattern is visible in this algorithms. INstead in the other algorithms the pattern is less visibile e secondo silvia per esempio il variance reduced non presenta un pattern appunto perche il noise e` piu distribuito siccome per la proprieta della varianza considerata nell'algoritmo.
Un piccolo confronto con il guassian noise come fatto nel jupyter sarebbe il top.
}
\begin{figure}[htbp]
	\centering
	\includegraphics[width=3.8cm]{image_perturbation_example_30.png}\hfil
	\includegraphics[width=3.8cm]{image_perturbation_example_35.png}
	\caption{Decentralized FW Perturbated Images: no clue on the parameters}
	\label{fig:decentralized}
\end{figure}
In Figure \ref{fig:decentralized} we can see an example of the perturbed images.

\subsection{Decentralized Variance-Reduced Stochastic Gradient Free Frank Wolfe}
For the Variance-Reduced FW algorithm we consider 5 workers and with 800 different images each, i.e. 160 images per digit. We set the number of queries to 20 and the number of component functions $S_2 = 3$. We then run the Algorithm \ref{variance-reduced} for $q=5,7,9$ and $n=5,10$. The choice for the values of $q$ is because of the different number of calling to the KWSA, while the $n$ parameter identify the different number of component function.\\
To quantify the performance of Algorithm \ref{variance-reduced}, it can be used the Frank-Wolfe duality gap, which is an upper bound on the primal suboptimality $f(x_t)-f(x^*)$, define as
\[ \mathcal{G} = \max_{v \in C} <F(x),x-v> \]
where $C$ is the associated constraint. In addition, it can also be used as a stopping criterion for FW algorithms.\\
From the theory we know that the major improvement of the variance reduction scheme is in terms of the fradient tracking performance, thath is, with $S_2 = (2d+9)\sqrt{n}/n_0$, $q = n_0 \sqrt{n}/6$, where $n_0 \in [1, \sqrt{n}/6)$. For the data that we have this tecnique for measuring the performances is unfeasible.\\

DA SPIEGARE MEGLIO, QUESTA PARTE NON L'AVEVO CAPITA MOLTO BENE :(

\subsection{Distributed Stochastic Gradient Free Frank Wolfe}
To test the performance of Algorithm \ref{distributed} we used 10 workers and an adjacecy matrix $A$ given by 
\[ A = 
\begin{pmatrix}
1& 1& 0& 1& 1& 1& 1& 1& 0& 1\\
1& 1& 1& 0& 1& 1& 1& 0& 1& 1\\
0& 1& 1& 1& 1& 1& 0& 1& 1& 1\\
1& 0& 1& 1& 1& 1& 0& 1& 1& 1\\
1& 1& 1& 1& 1& 1& 1& 0& 1& 1\\
1& 1& 1& 1& 1& 1& 1& 1& 1& 0\\
1& 1& 0& 0& 1& 1& 1& 1& 1& 1\\
1& 0& 1& 1& 0& 1& 1& 1& 1& 1\\
0& 1& 1& 1& 1& 1& 1& 1& 1& 1\\
1& 1& 1& 1& 1& 0& 1& 1& 1& 1	
\end{pmatrix}
.\]
We can notice that the diagonal is of ones, this because each node is connected to itself. Our network is composed of 10 nodes and the connectivity of the graph can be know by computing $\Vert W- J \Vert$, where $J= 11^T/10$ and $11^T$ represent a matrix with all entries set to 1. In our case we have a connectivity value of 0.438. We use 15 directions and test the algorithm for 20, 50 and 100 queries.

\begin{figure}[htbp]
	\centering
	\includegraphics[width=3cm]{report_distributed_delta_100_15.png}\hfil
	\includegraphics[width=5cm]{report_perturbated_img_100_15.png}
	\caption{Perturbation and perturbed image of the distributed FW with T=100 and m=15.}
	\label{fig:distributed_delta_50+20}
\end{figure}
In Figure \ref{fig:distributed_delta_50+20} we can see an example of the perturbations.
\subsection{Comparison between the perturbations}

\subsection{Decentralized SGF FW Experiments}
\textcolor{blue}{Usiamo il passato ora, siccome e` un esperimento svolto}
To study the performance of Algorithm \ref{decentralized} we used the 10000 images in the MNIST test set, after normalizing them.\\ We split the digit images by giving 10 samples of each class to our 10 workers. In such way, we make each worker holding a hundred images. After tuning the hyperparameter $m$ of the number of direction, we discovered that $m=15$ was the good compromise between the computation time and the overall results. For each image we estimated its gradient using 20, 50 and 100 queries.

\textcolor{gray}{Accuracy error achieved, during the training of the model, the we need to compare the result achieved by testing the model on the whole test set.\\ Notice that the drop of the accuracy, hence the accuracy is discovered already at the 10 epoch/iteration of the algorithms. Comparing the three algorithm is almost the faster one, the competing algorithm in terms of speed convergence of the noise is the distributed.
\\ presence of the pattern in the noise when the accuracy is minimized, more is minimized more the noise the pattern is visible in this algorithms. INstead in the other algorithms the pattern is less visibile e secondo silvia per esempio il variance reduced non presenta un pattern appunto perche il noise e` piu distribuito siccome per la proprieta della varianza considerata nell'algoritmo.
Un piccolo confronto con il guassian noise come fatto nel jupyter sarebbe il top.
}
\begin{figure}[htbp]
	\centering
	\includegraphics[width=7cm]{image_pertub_T20_final.png}
	\caption{Image of 4 changed to 3 with the adversarial perturbation generated by the Decentralized Algorithm \ref{decentralized} with a query of 20 and 15 directions.}
	\label{fig:decentralized}
\end{figure}
In Figure \ref{fig:decentralized} we can see an example of the perturbation applyed on an image of the MNIST digits.

\subsection{Decentralized Variance-Reduced SGF FW Experiments}
The experiments performed with Algorithm \ref{variance-reduced} were conducted considering $M=5$ workers. Each worker was fed with $S_1=800$ images (80 images per class) from the portion of the MNIST test set that LeNet5 was able to classify correctly. We imposed that the same image could not be assigned to different workers. This aspect was not clearly specified in the original algorithm proposed in \cite{A3}, section VI, in which the authors seem to suggest to use the same $S_1$ images for all the workers. However, the distributed data settings arise from the need of dividing huge datasets into different machines, for simplifying the computational complexity, and therefore we opted for an implementation choice that resulted more coherent with this idea.\\  The number $M$ of workers was halved compared to the one of the previous algorithm and also the other hyperparameters was chosen to be pretty low to reduce a bit the CPU-time consumption. In particular, we set the number of component functions to $n=5$ or $n=10$, the number of sampled components in RDSA to $S_2=3$, the number of queries to $T=20$ and the period parameter to $q=5$ or $q=7$ or $q=9$. With these settings, the algorithm approximatively took between one and four hours to terminate. In fact, the algorithm combines the hungry but accurate queries of KWSA, regulated by the parameter $q$, with the efficient but potentially inaccurate queries of RDSA. 
\begin{figure}[htbp]
	\centering
	\includegraphics[width=7cm]{image_pertub_q5_n10_final.png}
	\caption{Image of 1 changed to 4 with the adversarial perturbation generated by the Decentralized Variance-Reduced Algorithm \ref{variance-reduced} with q=5 and n=10.}
	\label{fig:variance-reduced}
\end{figure}
In Figure \ref{fig:variance-reduced} we can see an example of the perturbations.

\subsection{Distributed SGF FW Experiments}
To test the performance of Algorithm \ref{distributed} we used a distributed network of $M=10$ worker nodes and we
gave them 10 images per class each, for a total of 100 different images each.

The connectivity of the graph describing the distributed setting is represented by the value $\Vert \mathbf{W}- \mathbf{J} \Vert$,
where $\mathbf{J}= \mathbf{11}^T/M$ with $\mathbf{11}^T$ the matrix having all entries set to 1.\\ For the experiments, we created a network
with a connectivity value of 0.438 given by the following adjacecy
matrix:
\[ A =
\begin{pmatrix}
1& 1& 0& 1& 1& 1& 1& 1& 0& 1\\
1& 1& 1& 0& 1& 1& 1& 0& 1& 1\\
0& 1& 1& 1& 1& 1& 0& 1& 1& 1\\
1& 0& 1& 1& 1& 1& 0& 1& 1& 1\\
1& 1& 1& 1& 1& 1& 1& 0& 1& 1\\
1& 1& 1& 1& 1& 1& 1& 1& 1& 0\\
1& 1& 0& 0& 1& 1& 1& 1& 1& 1\\
1& 0& 1& 1& 0& 1& 1& 1& 1& 1\\
0& 1& 1& 1& 1& 1& 1& 1& 1& 1\\
1& 1& 1& 1& 1& 0& 1& 1& 1& 1
\end{pmatrix}
.\]
Since each node is connected to itself, the diagonal of the matrix A is filled with ones.
Furthermore, we computed the I-RDSA approximated gradient scheme along $m=15$ directions and we tested the algorithm for $T=20$, $T=50$ and $T=100$ queries.

In Figure \ref{fig:perturbations} we can see the universal adversarial perturbations produced by Algorithm \ref{distributed}
for different values of T. It's clear that the three perturbations are very similar to each other and all of them present a quite evident 3-shape.
\begin{figure}[h]
	\centering
	\begin{subfigure}[b]{0.15\textwidth}
		\centering
		\includegraphics[width=2.3cm]{T20_final_distr.png}
		\caption{}
		\label{fig:distributed_perturbation_20}
	\end{subfigure}
	\hfill
	\begin{subfigure}[b]{0.15\textwidth}
		\includegraphics[width=2.3cm]{T50_final_distr.png}
		\caption{}
		\label{fig:variance-distributed_perturbation_50}
	\end{subfigure}
	\hfill
	\begin{subfigure}[b]{0.15\textwidth}
		\includegraphics[width=2.9cm]{T100_final_bar.png}
		\caption{}
		\label{fig:distributed_perturbation_100}
	\end{subfigure}
	\caption{{\small Perturbations of the Distributed SGF FW algorithm created for different values of T:
	  \ref{fig:distributed_perturbation_20} for T=20, \ref{fig:variance-distributed_perturbation_50} for T=50, \ref{fig:distributed_perturbation_100} for T=100.}}
	\label{fig:perturbations}
\end{figure}


In Figure \ref{fig:distributed} is represented the perturbation of Figure \ref{fig:perturbations} (c) applied on an image from class 2.

\begin{figure}[htbp]
	\centering
	\includegraphics[width=6cm]{image_pertub_T100_final_distr.png}
	\caption{{\small Image belonging to class 2 but classified as 3 by using the adversarial perturbation in Figure \ref{fig:perturbations} (c) generated by the Distributed SGF FW method.}}
	\label{fig:distributed}
\end{figure}
Although we can still recognize the 3-shape pattern in the
perturbation, it is much smoother compared to the perturbation in Figure \ref{fig:decentralized_perturbations} (c) computed by the Decentralized
SGF FW algorithm. Nevertheless, it is strong enough to fool the classifier and lead it to a wrong prediction.

The Table \ref{tab:distributed} displays the values of accuracy reached by the LeNet-5 classifier on perturbed images, generated using the results of Algorithm \ref{decentralized} with different amounts of queries $T$.\\
\begin{table}[htbp]
	\begin{center}
		\begin{adjustwidth}{-.6cm}{}
			\begin{tabular}{c|ccc}
				\textbf{Attack} &          20 \textbf{queries} &      50 \textbf{queries} &     100 \textbf{queries} \\
				\midrule
				{\small Distributed SGF FW}     &   73.47\% &    74.62\% &       75.84\% \\
			\end{tabular}
		\end{adjustwidth}
	\end{center}
	\caption{{\small Summary of $\ell_\infty$ Universal Adversarial Perturbation with $\varepsilon$=0.25. MNIST attacks using Distributed SGF FW. The entries of the table represent the accuracies of LeNet-5 for the three different attacks.}}
	\label{tab:distributed}
\end{table}

We can see that there's no significant
difference among the accuracy values corresponding to different number of queries. What we can notice is that the
perturbation becomes slightly less effective in fooling the classifier, as long as we increase the number of queries.
\subsection{Generalization across Deep Neural Networks}

\section{Conclusions}
In this report we have focused on the problem of producing universal adversarial perturbations by analizing three
Stochastic Gradient Free Frank-Wolfe algorithms. In particular, we have shown that the perturbations created by
Decentralized and Distributed SGF FW algorithms follow a similar and more clear pattern compared to the Decentralized
Variance-Reduced SGF FW algorithm. In particular, we can clearly see that the reproduced pattern has a 3 shape, which
leads the majority of handwritten digits to be misclassified as 3. This can be explained by the concept of \textit{dominant labels},
mentioned in Section \ref{section:perturb}. In fact, number 3 is a wide number, that covers most of the space in the image. Therefore, a
perturbation with a 3 shape can easily lead to the misclassification of smaller numbers such as 1 and 7, which occupy
less space in the image. On the contrary, the perturbations produced by the Decentralized Variance-Reduced SGF FW algorithm,
don't have a clear pattern and the noise associated with them looks randomly spread.

Furthermore, the algorithm that reached better results in terms of misclassification is Algorithm \ref{decentralized},
which lowered the classifier's accuracy to 55\%. In this sense, the worst algorithm was \ref{variance-reduced} since
it was unable to lower the classifier's accuracy below 84\%.


% confronto tra i nostri metodi:
% - confronto pattern --> how the noise is spread in the perturbation
% - confronto accuracy --> small accuracy, best algorithm
% - confronto running-time?

{\small
	\bibliographystyle{ieee_fullname}
	\bibliography{egbib}
}
\end{document}
